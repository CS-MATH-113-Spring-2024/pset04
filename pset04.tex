\documentclass[a4paper]{exam}

\usepackage{amsmath,amsthm}
\usepackage{array}
\usepackage[a4paper]{geometry}

\header{CS/MATH 113}{PSet 04: Predicates, Quantifiers, and Nested Quantifiers}{Spring 2024}
\footer{}{Page \thepage\ of \numpages}{}
\runningheadrule
\runningfootrule

\title{Problem Set 04: Predicates, Quantifiers, and Nested Quantifiers}
\author{CS/MATH 113 Discrete Mathematics}
\date{Spring 2024}

\printanswers

\begin{document}
\maketitle

The problems below are grouped by the section of the book that they draw from. Many of them are similar to the worked examples in the section. Please consult the section for help if needed. You can also approach the course staff during their consultation hours are listed on Canvas.

Discussing the problems with your peers is encouraged and does not violate academic honesty unless the submitted solutions are highly similar.

\begin{questions}
\fullwidth{\section*{1.4 Predicates and Quantifiers}}
  % \question Determine the truth value of each of these statements if the domain of each variable consists of all real numbers.
  % \begin{parts}
  %   \part $\exists x (x^2 = 2)$
  %   \part $\exists x (x^2 = -1)$
  %   \part $\forall x (x^2+2\ge 1)$
  %   \part $\forall x (x^2\ne x)$
  % \end{parts}

  \question Express the negation of each of these statements in terms of quantifiers without using the negation symbol.
  \begin{parts}
    \part $\forall x (-2 < x < 3)$
    \begin{solution}
      $\exists x (x\le-2 \lor x\ge3)$
    \end{solution}
    \part $\forall x (0 \le x < 5)$
    \begin{solution}
      $\exists x (x<0 \lor x\ge5)$
    \end{solution}
    \part $\exists x (-4 \le x \le 1)$
    \begin{solution}
      $\forall x (x<-4 \lor x>1)$
    \end{solution}
    \part $\exists x (-5 < x < -1)$
    \begin{solution}
      $\forall x (x\le-5 \lor x\ge-1)$
    \end{solution}
  \end{parts}

  % \question Find a counterexample, if possible, to these universally quantified statements, where the domain for all variables consists of all real numbers.
  % \begin{parts}
  %   \part $\forall x (x^2 = x)$
  %   \part $\forall x (x^2 \ne 2)$
  %   \part $\forall x (|x| > 0)$
  % \end{parts}

  \question Determine whether the following pairs of statements are logically equivalent.
  \begin{parts}
    \part $\forall x (P(x) \implies Q(x))$ and $\forall x P(x) \implies \forall x Q(x)$
    \begin{solution}
      For the two statements to be logically equivalent, they must have the same truth values over all domains and predicates. We show that they are not equivalent by presenting a counterexample.

      \begin{proof}
      Consider a $P(x)$ that is true for some and false for other values in the domain.\\
      Consider a $Q(x)$ that is false for all values in the domain.

      LHS:\\
      $P(x)\implies Q(x)$ is true for some and false for other values in the domain.\\
      Therefore, LHS, i.e., $\forall x(P(x)\implies Q(x))$, is false.

      RHS:\\
      $\forall x P(x)$ is false. $\forall x Q(x)$ is false.\\
      Therefore RHS, i.e., $\forall x P(x) \implies \forall x Q(x)$, is true.
    \end{proof}
    \end{solution}
    \part $\forall x (P(x) \iff Q(x))$ and $\forall x P(x) \iff \forall x Q(x)$
    \begin{solution}
      For the two statements to be logically equivalent, they must have the same truth values over all domains and predicates. We show that they are not equivalent by presenting a counterexample.

      \begin{proof}
      Consider a $P(x)$ that is true for some and false for other values in the domain.\\
      Consider a $Q(x)$ that is false for all values in the domain.

      LHS:\\
      $P(x)\iff Q(x)$ is true for some and false for other values in the domain.\\
      Therefore, LHS, i.e., $\forall x(P(x)\iff Q(x))$, is false.

      RHS:\\
      $\forall x P(x)$ is false. $\forall x Q(x)$ is false.\\
      Therefore RHS, i.e., $\forall x P(x) \iff \forall x Q(x)$, is true.
    \end{proof}
  \end{solution}
  \end{parts}

  \question Show that the following pairs of statements are not logically equivalent.
  \begin{parts}
    \part $\forall x P(x) \lor \forall x Q(x)$ and $ \forall x (P(x) \lor Q(x))$
    \begin{solution}
      For the two statements to be logically equivalent, they must have the same truth values over all domains and predicates. We show that they are not equivalent by presenting a counterexample.

      \begin{proof}
      Consider a $P(x)$ that is true for some and false for other values in the domain.\\
      Consider $Q(x)\equiv\lnot P(x)$.
        
      LHS:\\
      $\forall x P(x)$ is false. $\forall x Q(x)$ is false.\\
      Therefore, LHS, i.e., $\forall x(P(x)\lor\forall x Q(x))$, is false.

      RHS:\\
      $P(x)\lor Q(x)$ is true for every value in the domain.\\
      Therefore RHS, i.e., $\forall x (P(x) \lor Q(x))$, is true.
      \end{proof}
    \end{solution}
    \part $\exists x P(x) \land \exists x Q(x)$ and $ \exists x (P(x) \land Q(x))$
    \begin{solution}
      For the two statements to be logically equivalent, they must have the same truth values over all domains and predicates. We show that they are not equivalent by presenting a counterexample.

      \begin{proof}
      Consider a $P(x)$ that is true for some values in the domain.\\
      Consider a $Q(x)$ that is true for some different values in the domain.
        
      LHS:\\
      $\exists x P(x)$ is true. $\exists x Q(x)$ is true.\\
      Therefore, LHS, i.e., $\exists x P(x)\land \exists x Q(x)$, is true.

      RHS:\\
      $P(x)\land Q(x)$ is false for every value in the domain.\\
      Therefore RHS, i.e., $\exists x (P(x) \land Q(x))$, is false.
      \end{proof}
    \end{solution}
  \end{parts}


\fullwidth{\section*{1.5 Nested Quantifiers}}

  \question A discrete mathematics class contains 1 CND freshman, 3 CND sophomores, 15 CS sophomores, 2 CND juniors, 2 CS juniors, and 1 CS senior. Express each of these statements in terms of quantifiers and then determine its truth value.
  \begin{parts}
    \part There is a student in the class who is a junior.
    \begin{solution} For this and subsequent parts, let
      $M(x,y): x$ has major $y$, where\\
      the domain of $x$ is all the students in the class, and the domain of $y$ is all majors,\\
      $R(x,y): x$ has rank $y$, where\\
      the domain of $x$ is all the students in the class, and the domain of $y$ is all ranks.

      $\exists x R(x, Junior)$\\
      True, e.g. any of the 2 CND juniors.
    \end{solution}
    \part Every student in the class is a CS major.
    \begin{solution}
      $\forall x M(x, CS)$\\
      False, e.g. the CND freshman.
    \end{solution}
    \part There is a student in the class who is neither a CND major nor a junior.
    \begin{solution}
      $\exists x (\lnot M(x, CND) \land \lnot R(x, Junior))$\\
      True, e.g. any of the 15 CS sophomores.
    \end{solution}
    \part Every student in the class is either a sophomore or a CS major.
    \begin{solution}
      $\forall x R(x,Sophomore) \lor M(x, CS))$\\
      False, e.g. any of the 2 CND juniors.
    \end{solution}
    \part There is a major such that there is a student in the class in every year of study with that major.
    \begin{solution}
      $\exists m\forall r\exists x M(x,m) \land R(x,r)$\\
      False. Students from only 2 majors are in the class. Of these, there is no CS freshman or CND senior in the class.
    \end{solution}
  \end{parts}
  
  \question Use predicates, quantifiers, logical connectives, and mathematical operators to express the statement that there is a positive integer that is not the sum of three squares.
    \begin{solution}
      Let $P(x) : x$ is a positive integer.\\
      $\exists x [P(x) \land \forall i\forall j\forall k (x\ne i^2 + j^2 + k^2)]$
    \end{solution}

  \question Determine the truth value of each of these statements if the domain of each variable consists of all real numbers. Where possible, provide an example if the statement is true, or a counterexample if the statement is False.
  \begin{parts}
    \part $\forall x \exists y (x^2 = y)$
    \begin{solution}
      True, every real number, $x$, has a square.
    \end{solution}
    \part $\forall x \exists y (x = y^2)$
    \begin{solution}
      False, any negative real number is a counter example for $x$. There is no $y$ such that $y^2=x$.
    \end{solution}
    \part $\exists x \forall y (xy = 0)$
    \begin{solution}
      True, e.g. $x=0$.
    \end{solution}
    \part $\exists x \exists y (x + y \ne y + x)$
    \begin{solution}
      False.
    \end{solution}
    \part $\forall x (x \ne 0 \implies \exists y (xy = 1))$
    \begin{solution}
      True, every non-zero real number has a multiplicative inverse.
    \end{solution}
    \part $\exists x \forall y (y \ne 0 \implies xy = 1)$
    \begin{solution}
      False, no real number multiplies with every real number to yield 1.
    \end{solution}
    \part $\forall x \exists y (x + y = 1)$
    \begin{solution}
      True, $y=1-x$.
    \end{solution}
    \part $\exists x \exists y (x + 2y = 2 \land 2x + 4y = 5)$
    \begin{solution}
      False, the simultaneous equations have no solution.
    \end{solution}
    \part $\forall x \exists y (x + y = 2 \land 2x - y = 1)$
    \begin{solution}
      False, there is a unique solution, $(x,y)=(1,1)$. The statement does not hold for all $x$.
    \end{solution}
    \part $\forall x \forall y \exists z (z = \frac{x + y}{2})$
    \begin{solution}
      True, every pair of real numbers has an average.
    \end{solution}
  \end{parts}

  \question Suppose the domain of the propositional function P(x, y) consists of pairs $x$ and $y$, where $x$ is $1,2,$ or $3$ and $y$ is $1,2,$ or $3$. Write out these propositions using disjunctions and conjunctions.
  \begin{parts}
    \part $\forall x \forall y P(x, y)$
    \begin{solution}
      $P(1,1) \land P(1,2) \land P(1,3) \land P(2,1) \land P(2,2) \land P(2,3) \land P(3,1) \land P(3,2) \land P(3,3)$
    \end{solution}
    \part $\exists x \exists y P(x, y)$
    \begin{solution}
      $P(1,1) \lor P(1,2) \lor P(1,3) \lor P(2,1) \lor P(2,2) \lor P(2,3) \lor P(3,1) \lor P(3,2) \lor P(3,3)$
    \end{solution}
    \part $\exists x \forall y P(x, y)$
    \begin{solution}
      $[P(1,1) \land P(1,2) \land P(1,3)] \lor [P(2,1) \land P(2,2) \land P(2,3)] \lor [P(3,1) \land P(3,2) \land P(3,3)]$
    \end{solution}
    \part $\forall x \exists y P(x, y)$
    \begin{solution}
      $[P(1,1) \lor P(1,2) \lor P(1,3)] \land [P(2,1) \lor P(2,2) \lor P(2,3)] \land [P(3,1) \lor P(3,2) \lor P(3,3)]$
    \end{solution}
  \end{parts}
  
  \question Find a counterexample, if possible, to these universally quantified statements, where the domain for all variables consists of all integers.
  \begin{parts}
    \part $\forall x \exists y (x = \frac{1}{y})$
    \begin{solution}
      One counterexample is $x=2$. No $y$ from the domain can satisfy the statement, $x = \frac{1}{y}$, when $x=2$.
    \end{solution}
    \part $\forall x \exists y (y^2 - x < 100)$
    \begin{solution}
      One counterexample is $x=-100$. No $y$ from the domain can satisfy the statement,$y^2 - x < 100$, when $x=-100$.
    \end{solution}
    \part $\forall x \forall y (x^2 \ne y^3)$
    \begin{solution}
      One counterexample is $(x,y)=(1,1)$. It does not satisfy the claim $x^2 \ne y^3$.
    \end{solution}
  \end{parts}

  \question Determine the truth value of the statement $\exists x \forall y (x \le y^2)$ if the common domain for the variables is as given below. Where possible, provide an example if the statement is true, or a counterexample if the statement is False.
  \begin{parts}
    \part the positive real numbers.
    \begin{solution}
      False, no such $x$ exists.
    \end{solution}
    \part the integers.
    \begin{solution}
      True, $x=0$ is an example.
    \end{solution}
    \part the nonzero real numbers.
    \begin{solution}
      True, any negative $x$ is an example.
    \end{solution}
  \end{parts}
  
  \question Recall the \textit{uniqueness quantifier}, $\exists!$. Express the quantification $\exists!x P(x)$ using universal and existential quantifiers, and logical operators.
    \begin{solution}
      $\exists!x P(x) \equiv \exists x [P(x) \land \forall y(P(y) \implies y=x)]$
    \end{solution}
  
\end{questions}
\end{document}
%%% Local Variables:
%%% mode: latex
%%% TeX-master: t
%%% End: